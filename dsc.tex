% This is "sig-alternate.tex" V2.0 May 2012
% This file should be compiled with V2.5 of "sig-alternate.cls" May 2012
%
% This example file demonstrates the use of the 'sig-alternate.cls'
% V2.5 LaTeX2e document class file. It is for those submitting
% articles to ACM Conference Proceedings WHO DO NOT WISH TO
% STRICTLY ADHERE TO THE SIGS (PUBS-BOARD-ENDORSED) STYLE.
% The 'sig-alternate.cls' file will produce a similar-looking,
% albeit, 'tighter' paper resulting in, invariably, fewer pages.
%
% ----------------------------------------------------------------------------------------------------------------
% This .tex file (and associated .cls V2.5) produces:
%       1) The Permission Statement
%       2) The Conference (location) Info information
%       3) The Copyright Line with ACM data
%       4) NO page numbers
%
% as against the acm_proc_article-sp.cls file which
% DOES NOT produce 1) thru' 3) above.
%
% Using 'sig-alternate.cls' you have control, however, from within
% the source .tex file, over both the CopyrightYear
% (defaulted to 200X) and the ACM Copyright Data
% (defaulted to X-XXXXX-XX-X/XX/XX).
% e.g.
% \CopyrightYear{2007} will cause 2007 to appear in the copyright line.
% \crdata{0-12345-67-8/90/12} will cause 0-12345-67-8/90/12 to appear in the copyright line.
%
% ---------------------------------------------------------------------------------------------------------------
% This .tex source is an example which *does* use
% the .bib file (from which the .bbl file % is produced).
% REMEMBER HOWEVER: After having produced the .bbl file,
% and prior to final submission, you *NEED* to 'insert'
% your .bbl file into your source .tex file so as to provide
% ONE 'self-contained' source file.
%
% ================= IF YOU HAVE QUESTIONS =======================
% Questions regarding the SIGS styles, SIGS policies and
% procedures, Conferences etc. should be sent to
% Adrienne Griscti (griscti@acm.org)
%
% Technical questions _only_ to
% Gerald Murray (murray@hq.acm.org)
% ===============================================================
%
% For tracking purposes - this is V2.0 - May 2012

\documentclass{sig-alternate}

\usepackage{graphicx}
\usepackage{datetime}
\newdateformat{mydate}{\THEDAY\ \monthname[\THEMONTH]\ \THEYEAR}

% arara: pdflatex 
% arara: bibtex 
% arara: pdflatex 
% arara: pdflatex 
\begin{document}

%
% --- Author Metadata here ---

\conferenceinfo{Data for Good Exchange 2015,}{New York, NY}

% \CopyrightYear{2007} % Allows default copyright year (20XX) to be over-ridden - IF NEED BE.
% \crdata{0-12345-67-8/90/01}  % Allows default copyright data (0-89791-88-6/97/05) to be over-ridden - IF NEED BE.
% --- End of Author Metadata ---

\title{Pushing Data Science Education into the Real World}
% \subtitle{xxxxxx}

%
% You need the command \numberofauthors to handle the 'placement
% and alignment' of the authors beneath the title.
%
% For aesthetic reasons, we recommend 'three authors at a time'
% i.e. three 'name/affiliation blocks' be placed beneath the title.
%
% NOTE: You are NOT restricted in how many 'rows' of
% "name/affiliations" may appear. We just ask that you restrict
% the number of 'columns' to three.
%
% Because of the available 'opening page real-estate'
% we ask you to refrain from putting more than six authors
% (two rows with three columns) beneath the article title.
% More than six makes the first-page appear very cluttered indeed.
%
% Use the \alignauthor commands to handle the names
% and affiliations for an 'aesthetic maximum' of six authors.
% Add names, affiliations, addresses for
% the seventh etc. author(s) as the argument for the
% \additionalauthors command.
% These 'additional authors' will be output/set for you
% without further effort on your part as the last section in
% the body of your article BEFORE References or any Appendices.

\numberofauthors{3}
%  in this sample file, there are a *total*
% of EIGHT authors. SIX appear on the 'first-page' (for formatting
% reasons) and the remaining two appear in the \additionalauthors section.
%
\author{
% You can go ahead and credit any number of authors here,
% e.g. one 'row of three' or two rows (consisting of one row of three
% and a second row of one, two or three).
%
% The command \alignauthor (no curly braces needed) should
% precede each author name, affiliation/snail-mail address and
% e-mail address. Additionally, tag each line of
% affiliation/address with \affaddr, and tag the
% e-mail address with \email.
%
% 1st. author
\alignauthor
Daniel Turek\\
       \affaddr{\small{Berkeley Institute for Data Science}}\\
       \affaddr{\small{University of California, Berkeley}}\\
       \affaddr{\small{190 Doe Library, Berkeley, CA}}\\
       \email{\normalsize{\texttt{anthonysuen@berkeley.edu}}}
% 2nd. author
\alignauthor
Anthony Suen\\
       \affaddr{\small{Berkeley Institute for Data Science}}\\
       \affaddr{\small{University of California, Berkeley}}\\
       \affaddr{\small{190 Doe Library, Berkeley, CA}}\\
       \email{\normalsize{\texttt{dturek@berkeley.edu}}}
% 3rd. author
\alignauthor
Dav Clark\\
       \affaddr{\small{Berkeley Institute for Data Science}}\\
       \affaddr{\small{University of California, Berkeley}}\\
       \affaddr{\small{190 Doe Library, Berkeley, CA}}\\
       \email{\normalsize{\texttt{davclark@berkeley.edu}}}
}

% \date{30 July 1999}
\date{\mydate\today}


% A category with the (minimum) three required fields
% \category{H.4}{Information Systems Applications}{Miscellaneous}
% A category including the fourth, optional field follows...
% \category{D.2.8}{Software Engineering}{Metrics}[complexity measures, performance measures]
% \terms{Theory}
% \keywords{ACM proceedings, \LaTeX, text tagging}


\maketitle

\begin{abstract}

The discipline of data science has been viewed as an convergence of high-power computing, data visualization and analysis, and data-driven application domains over the past decade.  Prominent research institutions and private sector industry have been quick to embrace the importance of data science, but the foundations for effective tertiary-level data science education are conspicuously absent. This is nothing new, however, as the university has a well-established tradition of developing its educational mission hand in hand with the development of novel methods for human understanding \cite{feingold_tradition_1991}. Thus it is natural that universities "figure out" data science hand in hand with the development of needed pedagogy. We consider the development of data science education with respect to recent trends in interdisciplinary and experiential education, along with agile and design thinking methodologies to understand how they could apply to data science educational programs. This historical perspective motivates us to consider what factors are necessary to drive effective data science education, which range from a complete end-to-end workflow, technological tools for development and team communications, and appropriate motivation and incentives. The first iteration of the \emph{Berkeley Institute for Data Science (BIDS) Collaborative} started in the University of California, Berkeley in the Spring of 2015 is used as a case study. From this we draw lessons learned and form a hypothesis regarding the necessary ingredients for effective data science education at the tertiary level ? a topic that is presently understudied.  This hypothesis will be tested and revised in subsequent iterations of the BIDS Collaborative as we continue our study of effective data science education, research, and social impact.

\end{abstract}

\section{Introduction}

The rapid advances in computational power and the on-going "big data" craze, the discipline of data science has exploded onto the academic and business landscape. Master's programs in data science are now being offered at premier research institutions such as Stanford University and Columbia University, and centers for data science have recently opened their doors at the University of California, Berkeley, the University of Washington, and New York University. Led  by the success of tech giants such as Google, Amazon and Facebook, the increasing availability of data is transforming industries ranging from medicine to media.  The industrial sector is keeping pace by creating and actively recruiting for positions in data science.  The profession of data scientist was even described by the Harvard Business Review as "the sexiest job of the 21$^{\text{st}}$ century" \cite{Patil2012}.

Despite this inundation of the term "data science," we still struggle to define what data science is, or to realise any boundaries as to what data science encompasses \cite{Hayashi1998, Loukides2011, Provost2013}. A common Venn diagram places data science squarely at the intersection of computer science, mathematical statistics, and scientific application domains.  This perhaps most accurately depicts that data science is nebulous by nature, having ties to all areas of quantitative scientific research or computational data analysis, but falls short of providing an understanding of how this new scientific discipline will eventually settle into the scientific ecosystem.  Fortunately, our aim is not to pin down the nature of data science itself, but instead to examine the practicalities and realities of data science education at the tertiary level.

There exists substantial literature regarding best practices and modern approaches to tertiary education. This has been a subject of interest since the first modern Universities appeared in Europe \cite{Rudy1984, Pedersen1997}.  Since then, the approach to higher education has evolved immensely, due to advances in technology, and also society's attitude towards higher education.  Perhaps the single-most transformative influence on higher education has been the so-called digital revolution of the past decades, which has had a profound impact on the content and style of tertiary education \cite{Roberts1994, Ely1995, Baker1997, Wood2005, Baek2008}.

Some research suggests that traditional approaches to tertiary education may only result in superficial learning, rather than a deep understanding of subject material \cite{Entwistle1992}.  Thus, the study of education itself is an area of prime interest.  Many approaches have been suggested and studied over the past decades, in attempts to improve education at the tertiary level.  \cite{Topping1996} promotes the practice of peer-tutoring, while others have more recently endorsed "flipped classrooms" in which learning becomes more self-directed (as opposed to instructor-directed) and classrooms become a place for practice instead of lecture \cite{Horn2013, Herreid2013}.  \cite{Ogawa1995} suggests a "mutiscience" approach to multidisciplinary science education, in which the diversity scientific disciplines is recognized and incorporated into the educational system.  Project-based learning has been promoted at the institutional level for many years \cite{Krajcik2006, Thomas2000}.

We re-focus our attention to the fledgling field of data science, but now in the context of education.  In light of the academic and industry spotlight on data science, experiences  and best practices for data science education should be a prime focus of research, just as it is for tertiary education in general.  However, owing to its relatively recent mainstream debut, there is a noteworthy absence of scientific research or published literature on the subject of data science education.  This fact motivates our present analysis of the history, current trends, and future prospects of data science education.

We aim to begin filling this void by providing a tangible case study of data science education, which was undertaken at the University of California, Berkeley, under the BIDS Collaborative. We consider the successes and failures of the first cohort to pass through the Collaborative, and the pain points which were encountered by the students and mentors, alike.  We make practical recommendations for educational approaches to data science curricula, and formulate a hypothesis regarding the "best practices" of tertiary data science education.  Study of our hypothesis will require subsequent experiential testing, which will be the subject of  on-going and future research.





\section{Paradigms of Data Science \\ Education}

To set the context for data science training and research methods, we briefly review several education and research paradigms that have become prominent in the past few decades. These include \textbf{interdisciplinary research}, \textbf{experiential learning}, and the \textbf{lean and agile practices} models of learning and research. We will also look at the Data Science for Social Good model (\texttt{http://dssg.io}) for training 40-50 graduate students in data science each summer since 2013. Finally, we will break down from these paradigms the theories we hope to practice in the BIDS Collaborative.

\subsection{Interdisciplinary Research}

Interdisciplinary or multidisciplinary research is "a format for conversation and connections that will lead to new knowledge" \cite{repko2008interdisciplinary}. The word has been fashionable academic buzzword for decades now, but it has looking and solving issues from multiple angles has failed to fundamentally scale beyond the confines of certain research groups to change the way research is carried out. There are major obstacles like cultural, organizational, technical barriers that prevent such learning and research environments \cite{eisenberg2000bridging}.

First, interdisciplinary training is not generally taught, whether you are an undergraduate or graduate student. Interdisciplinary is not a core metric for industry or academic career paths in order to be exposed to the values. This lack of training has to due in part to organizational structure with many departments providing little to no incentives for interdisciplinary collaborations. By making job prospects for those who have multidisciplinary focus difficult, it creates a vicious cycle that reinforces single disciplinary specialist work. 

Siloed organizational structures also created silos around sets of tools -- different software and methods are used to achieve similar goals via widely divergent means, potentially obscuring the fact that disparate groups are in fact grappling with the same underlying problems. Social scientists, physical sciences, and engineering use very different tools to tackle data. They use these different tools to run models that often have similar predictive goals. For example, Stata \cite{stata2005stata} might be used by an economist, Matlab\cite{incorporation2005matlab}  by the engineer, and R by the statistician. This divergences and specializations in tools creates an ever widening gap between major disciplines.  All of this makes interdisciplinary collaboration among a diverse team rare to come by.

Observing  these barriers to multidisciplinary data science, our mission was three fold. First, we implemented a framework for interdisciplinary learning beyond traditional academic lecture and coursework structures. Second, we cultivated an environment independent of the "rules" or established incentive structure of traditional academic departments. Our third goal was to show interdisciplinarity was possible with limited resources and incentives, and that data science tools can become unified. 

This reinforces a need for leaders within multidisciplinary teams which might be easier to achieve in a graduate student and undergraduate student teams than among faculty due to lower barriers in terms of technology, different incentivizes, and greater openness to doing things in a different way. We hypothesized that multidisciplinary collaboration would be easier when the stakes are smaller than traditional academics and learning is driven within student peer groups.

\subsection{Experiential Learning \\(Flipped Classroom Model)}

The second key component is learning real projects with real clients. Research from Stanford \cite{plotnikoff_classes_2013} has validated that the flipped class experiential learning as a stronger learning strategy compared to homework and lectures. Pushing students to do work without first preparation seems to be effective in accelerating knowledge. This meant we were scraping the lecture and testing format entirely. Experiential learning also emphasis project management since classroom projects are have clear deadlines scheduled by the professor \cite{mok2014teaching}. Given the diverse nature of each project, our limited staffing resources, and the need to manage relationships with clients, it became critical that students figure become project managers to push the team forward to success.

\subsection{Lean and Agile Practices}

Another method we wanted to adopt for our projects were Lean and Agile. Lean comes from lean manufacturing which seeks to deliver speed, quality, and alignment. Analytics in organizations is usually done using a waterfall method which require long development cycles. Lean has a great way to look at help accelerate operations, m operates as a whole.

\begin{figure*}
\centerline{\includegraphics[scale=0.3]{dsc_figure_gitcommits.png}}
\caption{CAPTION}
% \label{LABEL}
\end{figure*}


\section{Lessons Learned}

The first iteration of the BIDS Collaborative was an experimental undertaking, in which both students and facilitators were learning through the duration of the program.  Subsequent to this, many important lessons for effective data science education became apparent.  These lessons have been documented, and will serve as guiding principles for subsequent iterations of the Collaborative. We now discuss several of these "lessons learned," which will lead to our hypothesis for achieving effective data science education.

Using a model where students decide among various pre-determined capstone research projects, it is important to present options representing a broad variety of disciplines.  For example, having projects relating to physical sciences, social sciences, technology, health, environment, commerce, among many other possibilities.  By presenting this diverse range of project options, we were able to leverage students' innate interests in particular areas of study; thus, students did not feel shoehorned into a research areas of little or no interest to them.

It was critical that the logistics of each project were fully in order in advance of being presented to students.  Projects must be well-posed, and have a well-defined goal which students could work towards.  This also included having a responsive and interested point of contact, representing the client or the underlying organization, who could answer questions as necessary.  And most importantly, the relevant data must be available in advance, such that students could get a sense of what the project would entail, and could begin work immediately.  This consideration encompasses any legal releases, non-disclosure agreements, etc, pertaining to data access.  Fundamentally, for projects to begin with an enthusiastic start, students must have adequate access to the necessary data and a contact point representing the underlying organization.

Project and team components needed to be organized in the appropriate order, as each step necessarily followed the previous.  This process began with identifying clients with data science research projects of the appropriate scale.  The relevant data must be already available, as this was critical to generate student interest.  Next, we were able to identify student team leads, who had the motivation and skills to lead a research team, and interact with the client.  The team leads played a critical role interacting between these two parties, and having this team structure in place from the beginning was important for the overall organization of each team.  Finally, with all these parts in place, we were able to assemble teams for each project based on students' interests and skill sets.  Only by following this order of client, data, team lead, team members, did each step flow smoothly from those previous.

Along these same lines, we observed that regular interaction between the client and team lead was necessary. The team lead served as the bridge, for processing and presenting the client needs to the team of student researchers.  We learned it would not be possible the facilitators of the BIDS Collaborative to fill this role for several reasons.  First, it was impractical to micro-manage each project at this level.  And second, requiring this direct interaction between clients and team leads formed the reality of students organizing and accomplishing a real-world project.  This effectively maintained a sense of responsibility and self accomplishment for team leaders and team members, alike.

As projects got underway, we observed that creating well-defined intermediate goals was helpful to maintain forward momentum within each team.  These took for the form of "milestones" which would otherwise exist in a data science research workflow (e.g., data cleaning, or preliminary visualization analyses), but formally assigning dates for these tasks ensured that each team was progressing forward.  In addition, this helped the Collaborative facilitators passively monitor the progress of individual teams, and provide additional help when necessary.  Larger milestones were also created including mid-semester presentations, and a final capstone event where projects and results were presented to the clients.  The existence of these formal milestones kept all research teams on forward-moving (and similar) timelines.

Finally, we noticed the importance of introducing (and promoting the use of) team collaborative tools from the very beginning.  The most successful project teams immediately adopted GitHub for all project code, and also Slack for team communications.  It appeared that the sooner team members adopted these tools into their research workflows, the sooner meaningful progress began.  We believe it is critical to introduce these tools to students not familiar with them immediately, and promote, if not mandate their usage for teamwork and communications.


\section{Closing Thoughts}

A number of open questions remain from the first iteration of the Collaborative.  We first discuss a few of these questions, before presenting our conclusions and hypothesis for future data science education.

\subsection{Open Questions}

We explored possibilities for offering academic credit as motivation for students, but no students decided to pursue this option. The reason behind this remains uncertain, though it could be due to the lack of structure and the additional hurdles in enrollment. We believe one pathway would be creating a framework that plugs into an existing or a new project-oriented course.  Even so, however, the option of receiving course credit did not appear to be a strong motivation for students.

Some fraction of students, however motivated, were not prepared with the technical background for jumping into a data science research project.  The usefulness of periodic training sessions, and how these could be organized or delivered, remains an open question.  Who would teach such sessions and exactly what material would be most beneficial for students is also unclear.  This training could possibly link with existing workshop or training programs on campus, so certain students have the opportunity to get up to speed on the relevant tools.

Generally, the best approach to organizing, managing, and motivating teams remains unclear to the facilitators.  One approach would be to micromanage to some degree, and organize regular weekly team meetings.  However, this level of management is very time-consuming, and not always effective or appreciated by student groups.  In addition, exactly how to motivate a strong commitment from team members is a difficult question.  Fundamentally, we would like to rely on students' desires for real-world data science experience and education, but this will not always suffice.  How all students can be effectively motivated remains open for discussion.

\subsection{Conclusions}

The BIDS Collaborative was a small educational experiment done in BIDS with a bare bones staffing and limited planning. We did not have the resources of the Data Science for Social Good but we were successful in motivating a group of students to complete client provided real-world data science problems over the course of an academic semester.  The design and overall success of the BIDS Collaborative program shows promise in terms of scalability among the wider university.

Stepping back, we have uncovered certain best practices in terms of creating multidisciplinary teams to solve real life challenges. 

[some statement about stepping back, and how our conclusions and lessons can be applied equally well to well to non-data-science projects, etc]

Hypothesis:  [NEEDS TO BE WRITTEN]
We hypothesize that?.
Effective multidisciplinary data science education must deal with the complexity that is fundamental to novel data sources and / or methods. This requires a unified academic curriculum, augmented with workshops and consulting services that can help teams develop compatible approaches for working together, Most importantly, we have reported on the apparent value of a capstone collaborative project that provides an opportunity for a multidisciplinary groups to integrate their skills.



%\subsection{Figures}
%Like tables, figures cannot be split across pages; the best placement for them is typically the top or the bottom of the page nearest their initial mention.  To ensure this proper ``floating'' placement of figures, use the environment \textbf{figure} to enclose the figure and its caption.
%
%This sample document contains examples of \textbf{.eps} and \textbf{.ps} files to be displayable with \LaTeX.  More details on each of these is found in the \textit{Author's Guide}.
%
%\begin{figure}[h]
%\centerline{\includegraphics[scale=1.0]{FIGNAME}}
%\caption{CAPTION}
%\label{LABEL}
%\end{figure}
% 
%\begin{figure}
%\centering
%\epsfig{file=fly.eps}
%\caption{A sample black and white graphic (.eps format).}
%\end{figure}
%
%\begin{figure}
%\centering
%\epsfig{file=fly.eps, height=1in, width=1in}
%\caption{A sample black and white graphic (.eps format)
%that has been resized with the \texttt{epsfig} command.}
%\end{figure}
%
%As was the case with tables, you may want a figure that spans two columns.  To do this, and still to ensure proper ``floating'' placement of tables, use the environment \textbf{figure*} to enclose the figure and its caption. and don't forget to end the environment with {figure*}, not {figure}!
%
%\begin{figure*}
%\centering
%\epsfig{file=flies.eps}
%\caption{A sample black and white graphic (.eps format)
%that needs to span two columns of text.}
%\end{figure*}
%
%Note that either {\textbf{.ps}} or {\textbf{.eps}} formats are used; use the \texttt{{\char'134}epsfig} or \texttt{{\char'134}psfig} commands as appropriate for the different file types.
%
%\begin{figure}
%\centering
%\psfig{file=rosette.ps, height=1in, width=1in,}
%\caption{A sample black and white graphic (.ps format) that has
%been resized with the \texttt{psfig} command.}
%\vskip -6pt
%\end{figure}



%ACKNOWLEDGMENTS are optional

%\section{Acknowledgments}
%This section is optional; it is a location for you to acknowledge grants, funding, editing assistance and what have you.  In the present case, for example, the authors would like to thank Gerald Murray of ACM for his help in codifying this \textit{Author's Guide} and the \textbf{.cls} and \textbf{.tex} files that it describes.



% The following two commands are all you need in the
% initial runs of your .tex file to
% produce the bibliography for the citations in your paper.

\bibliographystyle{abbrv}
% \bibliography{sigproc}    % sigproc.bib is the name of the Bibliography in this case
\bibliography{ZoteroLibrary,davs-lone-reference}



% You must have a proper ".bib" file
%  and remember to run:
% latex bibtex latex latex
% to resolve all references
%
% ACM needs 'a single self-contained file'!



%%APPENDICES are optional

%%\balancecolumns
%\appendix
%%Appendix A
%\section{Headings in Appendices}
%The rules about hierarchical headings discussed above for
%the body of the article are different in the appendices.
%In the \textbf{appendix} environment, the command
%\textbf{section} is used to
%indicate the start of each Appendix, with alphabetic order
%designation (i.e. the first is A, the second B, etc.) and
%a title (if you include one).  So, if you need
%hierarchical structure
%\textit{within} an Appendix, start with \textbf{subsection} as the
%highest level. Here is an outline of the body of this
%document in Appendix-appropriate form:
%\subsection{Introduction}
%\subsection{The Body of the Paper}
%\subsubsection{Type Changes and  Special Characters}
%\subsubsection{Math Equations}
%\paragraph{Inline (In-text) Equations}
%\paragraph{Display Equations}
%\subsubsection{Citations}
%\subsubsection{Tables}
%\subsubsection{Figures}
%\subsubsection{Theorem-like Constructs}
%\subsubsection*{A Caveat for the \TeX\ Expert}
%\subsection{Conclusions}
%\subsection{Acknowledgments}
%\subsection{Additional Authors}
%This section is inserted by \LaTeX; you do not insert it.
%You just add the names and information in the
%\texttt{{\char'134}additionalauthors} command at the start
%of the document.
%\subsection{References}
%Generated by bibtex from your ~.bib file.  Run latex,
%then bibtex, then latex twice (to resolve references)
%to create the ~.bbl file.  Insert that ~.bbl file into
%the .tex source file and comment out
%the command \texttt{{\char'134}thebibliography}.
%% This next section command marks the start of
%% Appendix B, and does not continue the present hierarchy
%\section{More Help for the Hardy}
%The sig-alternate.cls file itself is chock-full of succinct
%and helpful comments.  If you consider yourself a moderately
%experienced to expert user of \LaTeX, you may find reading
%it useful but please remember not to change it.
%%\balancecolumns % GM June 2007
%% That's all folks!



\end{document}
