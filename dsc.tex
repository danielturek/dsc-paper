% This is "sig-alternate.tex" V2.0 May 2012
% This file should be compiled with V2.5 of "sig-alternate.cls" May 2012
%
% This example file demonstrates the use of the 'sig-alternate.cls'
% V2.5 LaTeX2e document class file. It is for those submitting
% articles to ACM Conference Proceedings WHO DO NOT WISH TO
% STRICTLY ADHERE TO THE SIGS (PUBS-BOARD-ENDORSED) STYLE.
% The 'sig-alternate.cls' file will produce a similar-looking,
% albeit, 'tighter' paper resulting in, invariably, fewer pages.
%
% ----------------------------------------------------------------------------------------------------------------
% This .tex file (and associated .cls V2.5) produces:
%       1) The Permission Statement
%       2) The Conference (location) Info information
%       3) The Copyright Line with ACM data
%       4) NO page numbers
%
% as against the acm_proc_article-sp.cls file which
% DOES NOT produce 1) thru' 3) above.
%
% Using 'sig-alternate.cls' you have control, however, from within
% the source .tex file, over both the CopyrightYear
% (defaulted to 200X) and the ACM Copyright Data
% (defaulted to X-XXXXX-XX-X/XX/XX).
% e.g.
% \CopyrightYear{2007} will cause 2007 to appear in the copyright line.
% \crdata{0-12345-67-8/90/12} will cause 0-12345-67-8/90/12 to appear in the copyright line.
%
% ---------------------------------------------------------------------------------------------------------------
% This .tex source is an example which *does* use
% the .bib file (from which the .bbl file % is produced).
% REMEMBER HOWEVER: After having produced the .bbl file,
% and prior to final submission, you *NEED* to 'insert'
% your .bbl file into your source .tex file so as to provide
% ONE 'self-contained' source file.
%
% ================= IF YOU HAVE QUESTIONS =======================
% Questions regarding the SIGS styles, SIGS policies and
% procedures, Conferences etc. should be sent to
% Adrienne Griscti (griscti@acm.org)
%
% Technical questions _only_ to
% Gerald Murray (murray@hq.acm.org)
% ===============================================================
%
% For tracking purposes - this is V2.0 - May 2012

\documentclass{sig-alternate}

\usepackage{datetime}
\newdateformat{mydate}{\THEDAY\ \monthname[\THEMONTH]\ \THEYEAR}

% arara: pdflatex 
% arara: bibtex 
% arara: pdflatex 
% arara: pdflatex 
\begin{document}

%
% --- Author Metadata here ---

\conferenceinfo{Data for Good Exchange 2015,}{New York, NY}

% \CopyrightYear{2007} % Allows default copyright year (20XX) to be over-ridden - IF NEED BE.
% \crdata{0-12345-67-8/90/01}  % Allows default copyright data (0-89791-88-6/97/05) to be over-ridden - IF NEED BE.
% --- End of Author Metadata ---

\title{Pushing Data Science Education into the Real World}
% \subtitle{xxxxxx}

%
% You need the command \numberofauthors to handle the 'placement
% and alignment' of the authors beneath the title.
%
% For aesthetic reasons, we recommend 'three authors at a time'
% i.e. three 'name/affiliation blocks' be placed beneath the title.
%
% NOTE: You are NOT restricted in how many 'rows' of
% "name/affiliations" may appear. We just ask that you restrict
% the number of 'columns' to three.
%
% Because of the available 'opening page real-estate'
% we ask you to refrain from putting more than six authors
% (two rows with three columns) beneath the article title.
% More than six makes the first-page appear very cluttered indeed.
%
% Use the \alignauthor commands to handle the names
% and affiliations for an 'aesthetic maximum' of six authors.
% Add names, affiliations, addresses for
% the seventh etc. author(s) as the argument for the
% \additionalauthors command.
% These 'additional authors' will be output/set for you
% without further effort on your part as the last section in
% the body of your article BEFORE References or any Appendices.

\numberofauthors{3}
%  in this sample file, there are a *total*
% of EIGHT authors. SIX appear on the 'first-page' (for formatting
% reasons) and the remaining two appear in the \additionalauthors section.
%
\author{
% You can go ahead and credit any number of authors here,
% e.g. one 'row of three' or two rows (consisting of one row of three
% and a second row of one, two or three).
%
% The command \alignauthor (no curly braces needed) should
% precede each author name, affiliation/snail-mail address and
% e-mail address. Additionally, tag each line of
% affiliation/address with \affaddr, and tag the
% e-mail address with \email.
%
% 1st. author
\alignauthor
Daniel Turek\\
       \affaddr{\small{Berkeley Institute for Data Science}}\\
       \affaddr{\small{University of California, Berkeley}}\\
       \affaddr{\small{190 Doe Library, Berkeley, CA}}\\
       \email{\normalsize{\texttt{anthonysuen@berkeley.edu}}}
% 2nd. author
\alignauthor
Anthony Suen\\
       \affaddr{\small{Berkeley Institute for Data Science}}\\
       \affaddr{\small{University of California, Berkeley}}\\
       \affaddr{\small{190 Doe Library, Berkeley, CA}}\\
       \email{\normalsize{\texttt{dturek@berkeley.edu}}}
% 3rd. author
\alignauthor
Dav Clark\\
       \affaddr{\small{Berkeley Institute for Data Science}}\\
       \affaddr{\small{University of California, Berkeley}}\\
       \affaddr{\small{190 Doe Library, Berkeley, CA}}\\
       \email{\normalsize{\texttt{davclark@berkeley.edu}}}
% \and  % use '\and' if you need 'another row' of author names
% 4th. author
%\alignauthor Lawrence P. Leipuner\\
%       \affaddr{Brookhaven Laboratories}\\
%       \affaddr{Brookhaven National Lab}\\
%       \affaddr{P.O. Box 5000}\\
%       \email{lleipuner@researchlabs.org}
%% 5th. author
%\alignauthor Sean Fogarty\\
%       \affaddr{NASA Ames Research Center}\\
%       \affaddr{Moffett Field}\\
%       \affaddr{California 94035}\\
%       \email{fogartys@amesres.org}
%% 6th. author
%\alignauthor Charles Palmer\\
%       \affaddr{Palmer Research Laboratories}\\
%       \affaddr{8600 Datapoint Drive}\\
%       \affaddr{San Antonio, Texas 78229}\\
%       \email{cpalmer@prl.com}
}   % Daniel's note: this closes the "\author{" above, need to keep it

% There's nothing stopping you putting the seventh, eighth, etc.
% author on the opening page (as the 'third row') but we ask,
% for aesthetic reasons that you place these 'additional authors'
% in the \additional authors block, viz.
% \additionalauthors{Additional authors: John Smith (The Th{\o}rv{\"a}ld Group,
% email: {\texttt{jsmith@affiliation.org}}) and Julius P.~Kumquat
% (The Kumquat Consortium, email: {\texttt{jpkumquat@consortium.net}}).}

% Just remember to make sure that the TOTAL number of authors
% is the number that will appear on the first page PLUS the
% number that will appear in the \additionalauthors section.

% \date{30 July 1999}
\date{\mydate\today}


% A category with the (minimum) three required fields
% \category{H.4}{Information Systems Applications}{Miscellaneous}
% A category including the fourth, optional field follows...
% \category{D.2.8}{Software Engineering}{Metrics}[complexity measures, performance measures]
% \terms{Theory}
% \keywords{ACM proceedings, \LaTeX, text tagging}


\maketitle

\begin{abstract}
The discipline of data science has been viewed as an convergence of high-power computing, data visualization and analysis, and data-driven application domains over the past decade.  Prominent research institutions and private sector industry have been quick to embrace the importance of data science, but the foundations for effective tertiary-level data science education are conspicuously absent. This is nothing new, however, as the university has a well-established tradition of developing its educational mission hand in hand with the development of novel methods for human understanding (Feingold, 1991). Thus it is natural that universities "figure out" data science hand in hand with the development of needed pedagogy. We consider the development of data science education with respect to recent trends in interdisciplinary and experiential education, along with agile and design thinking methodologies to understand how they could apply to data science educational programs. This historical perspective motivates us to consider what factors are necessary to drive effective data science education, which range from a complete end-to-end workflow, technological tools for development and team communications, and appropriate motivation and incentives. The first iteration of the \emph{Berkeley Institute for Data Science (BIDS)} Collaborative started in the University of California, Berkeley in the Spring of 2015 is used as a case study. From this we draw lessons learned and form a hypothesis regarding the necessary ingredients for effective data science education at the tertiary level.  This hypothesis will be tested and revised in subsequent iterations of the BIDS Collaborative as we continue our study of effective data science education, research, and social impact.
\end{abstract}


\section{Introduction}

Manuscript text here.

Testing citations \cite{burns_cambridge_2012, Gimenez2007, Hoffman2014}.


%\subsection{Figures}
%Like tables, figures cannot be split across pages; the best placement for them is typically the top or the bottom of the page nearest their initial mention.  To ensure this proper ``floating'' placement of figures, use the environment \textbf{figure} to enclose the figure and its caption.
%
%This sample document contains examples of \textbf{.eps} and \textbf{.ps} files to be displayable with \LaTeX.  More details on each of these is found in the \textit{Author's Guide}.
%
%\begin{figure}
%\centering
%\epsfig{file=fly.eps}
%\caption{A sample black and white graphic (.eps format).}
%\end{figure}
%
%\begin{figure}
%\centering
%\epsfig{file=fly.eps, height=1in, width=1in}
%\caption{A sample black and white graphic (.eps format)
%that has been resized with the \texttt{epsfig} command.}
%\end{figure}
%
%As was the case with tables, you may want a figure that spans two columns.  To do this, and still to ensure proper ``floating'' placement of tables, use the environment \textbf{figure*} to enclose the figure and its caption. and don't forget to end the environment with {figure*}, not {figure}!
%
%\begin{figure*}
%\centering
%\epsfig{file=flies.eps}
%\caption{A sample black and white graphic (.eps format)
%that needs to span two columns of text.}
%\end{figure*}
%
%Note that either {\textbf{.ps}} or {\textbf{.eps}} formats are used; use the \texttt{{\char'134}epsfig} or \texttt{{\char'134}psfig} commands as appropriate for the different file types.
%
%\begin{figure}
%\centering
%\psfig{file=rosette.ps, height=1in, width=1in,}
%\caption{A sample black and white graphic (.ps format) that has
%been resized with the \texttt{psfig} command.}
%\vskip -6pt
%\end{figure}



%ACKNOWLEDGMENTS are optional

%\section{Acknowledgments}
%This section is optional; it is a location for you to acknowledge grants, funding, editing assistance and what have you.  In the present case, for example, the authors would like to thank Gerald Murray of ACM for his help in codifying this \textit{Author's Guide} and the \textbf{.cls} and \textbf{.tex} files that it describes.



% The following two commands are all you need in the
% initial runs of your .tex file to
% produce the bibliography for the citations in your paper.

\bibliographystyle{abbrv}
% \bibliography{sigproc}    % sigproc.bib is the name of the Bibliography in this case
\bibliography{ZoteroLibrary.bib}



% You must have a proper ".bib" file
%  and remember to run:
% latex bibtex latex latex
% to resolve all references
%
% ACM needs 'a single self-contained file'!



%%APPENDICES are optional

%%\balancecolumns
%\appendix
%%Appendix A
%\section{Headings in Appendices}
%The rules about hierarchical headings discussed above for
%the body of the article are different in the appendices.
%In the \textbf{appendix} environment, the command
%\textbf{section} is used to
%indicate the start of each Appendix, with alphabetic order
%designation (i.e. the first is A, the second B, etc.) and
%a title (if you include one).  So, if you need
%hierarchical structure
%\textit{within} an Appendix, start with \textbf{subsection} as the
%highest level. Here is an outline of the body of this
%document in Appendix-appropriate form:
%\subsection{Introduction}
%\subsection{The Body of the Paper}
%\subsubsection{Type Changes and  Special Characters}
%\subsubsection{Math Equations}
%\paragraph{Inline (In-text) Equations}
%\paragraph{Display Equations}
%\subsubsection{Citations}
%\subsubsection{Tables}
%\subsubsection{Figures}
%\subsubsection{Theorem-like Constructs}
%\subsubsection*{A Caveat for the \TeX\ Expert}
%\subsection{Conclusions}
%\subsection{Acknowledgments}
%\subsection{Additional Authors}
%This section is inserted by \LaTeX; you do not insert it.
%You just add the names and information in the
%\texttt{{\char'134}additionalauthors} command at the start
%of the document.
%\subsection{References}
%Generated by bibtex from your ~.bib file.  Run latex,
%then bibtex, then latex twice (to resolve references)
%to create the ~.bbl file.  Insert that ~.bbl file into
%the .tex source file and comment out
%the command \texttt{{\char'134}thebibliography}.
%% This next section command marks the start of
%% Appendix B, and does not continue the present hierarchy
%\section{More Help for the Hardy}
%The sig-alternate.cls file itself is chock-full of succinct
%and helpful comments.  If you consider yourself a moderately
%experienced to expert user of \LaTeX, you may find reading
%it useful but please remember not to change it.
%%\balancecolumns % GM June 2007
%% That's all folks!



\end{document}
